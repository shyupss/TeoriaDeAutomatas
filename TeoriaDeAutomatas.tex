\documentclass[12pt]{article}
\usepackage[utf8]{inputenc}   % para tildes
\usepackage[spanish]{babel}   % español
\usepackage{color}            % colores para código
\usepackage{xcolor}           % colores en general
\definecolor{blueShyupss}{RGB}{0,0,128}
\usepackage[colorlinks=true, linkcolor=blueShyupss, urlcolor=gray, citecolor=orange]{hyperref} % para links (están con colores personalizados)
\usepackage{amsmath, amssymb} % símbolos y matemáticas
\usepackage{graphicx}         % imágenes
\usepackage{listings}         % código fuente
\usepackage{fancyhdr}         % encabezado/pie de página
\usepackage{geometry}         % márgenes
\usepackage{parskip}          % para los saltos

% definiciones para las llamadas a definiciones y teoremas
\newtheorem{teorema}{Teorema}
\newtheorem{definicion}{Definición}

% Para el texto de "Ejemplo:"
\newcommand{\ejemplo}{\vspace{1.1em}\textbf{\vspace{0.2em}\textit{Ejemplo:}} } 

% definicion para los "enunciados"
\newcommand{\enunciado}[1]{
  \vspace{1em} % Añade un pequeño espacio antes
  \noindent \textbf{\large #1} % Usamos \Large para el tamaño
  \vspace{1em} % Añade un pequeño espacio después
}

% definicion para los desarrollos
\newcommand{\desarrollo}[1]{
  \vspace{1em} % Añade un pequeño espacio antes
  \noindent \text{\large \sloppy \parbox{\textwidth}{#1}} % Ajusta el texto automáticamente con salto de línea
  \vspace{1em} % Añade un pequeño espacio después
}

\geometry{margin=2cm}

% inicio del documento
\begin{document}

\title{Teoría de Automatas}
\author{Diego Soto - Universidad Austral De Chile}
\date{\today}
\maketitle

\newpage %índice automático en nueva página
\tableofcontents
\newpage

% aquí comienza el contenido del shyupss, ordenado por secciones
\section{Introducción}

La teoría de autómatas constituye uno de los pilares fundamentales de la computación teórica y el análisis formal de lenguajes. Esta disciplina permite modelar y analizar el comportamiento de sistemas computacionales mediante representaciones abstractas conocidas como autómatas. A través del estudio de estos modelos, es posible comprender mejor cómo funcionan los lenguajes formales, los compiladores, y las máquinas que los procesan. 

Para comenzar, definiremos y repasaremos algunos conceptos que son previamente sabidos del ramo Estructuras Discretas y luego de a poco, nos iremos adentrando en materia.

\section{Alfabetos y Lenguajes}

Cuando hablamos comunmente de un lenguaje, se nos puede venir a la distintas ideas, como palabras, letras, etcétera. 
Comenzaremos dando algunas definiciones de estas, para poder empezar a contruir otras fokin definiciones.

\begin{definicion}
  Un Alfabeto es un conjunto finito de símbolos. \begin{center} $\Sigma = \{a, b, c, ..., z\}$\end{center}
\end{definicion}

\begin{definicion}
  Una Palabra es una secuencia ordenada de símbolos de un alfabeto. \par \begin{center} $x = a_1a_2a_3...a_4$ \end{center}
\end{definicion}

\begin{definicion}
  El largo de una palabra es el número de símbolos que los conforman. \par \begin{center}$\left\lvert x \right\rvert  = n$\end{center}
\end{definicion}

\begin{definicion}
  La palabra vacía es una palabra que no tiene ningún símbolo.
  \begin{center}
    Palabra vacía: $\varepsilon \Rightarrow \left\lvert \varepsilon \right\rvert = 0$
  \end{center}
\end{definicion}

\begin{definicion}
  Si $\Sigma$ es un alfabeto, anotaremos como $\Sigma^*$ el conjunto de todas las palabras posibles sobre $\Sigma$.
  \begin{center}
    $\Sigma^* = \bigcup_{k\geq0} \Sigma^k = \Sigma^0 \cup \Sigma^1 \cup \Sigma^2 \cup ...$

    En donde $\Sigma^k$ es el conjunto de palabras de largo k.
  \end{center}
\end{definicion}

\begin{ejemplo}
  \begin{center}
    Si $\Sigma = \left\{a, b\right\} $, entonces: 
  
    $\Sigma^0 = \left\{\epsilon\right\}$
  
    $\Sigma^1 = \left\{a, b\right\}$
  
    $\Sigma^2 = \left\{aa, ab, ba, bb\right\}$
  
    \dots
    
    $\Sigma^n = \left\{\epsilon, a, b, aa, ab, ba, bb, aaa, \dots\right\}$
  \end{center}
\end{ejemplo}

\begin{definicion}
  Un lenguaje es un conjunto de palabras que se pueden armar sobre un alfabeto $\Sigma$, el cual denotaremos por la letra $L$

  \begin{center}
    Con esto, tenemos que $L\subseteq \Sigma^*$
  \end{center}
\end{definicion}

\begin{ejemplo}
  \begin{itemize}
    \item $L = \{ x \in \{a, b\}^* \mid \text{x comienza con 'a'} \}$

    $L = \{a, aa, ab, aab, aba, \dots\}$
    \item $L = \{ x \in \{0, 1\}^* \mid \text{x es palindromo'} \}$
    
    $L = \{0, 1, 00, 11, 101, 010, 11011, 0110, \dots\}$
    \item $L = \{ x \in 6^+ \mid \text{Los dígitos de x suman 10}\}$
    
    $L = \{64, 22222, 235, \dots\} \text{ (Notar que las palabras acá son finitas)}$
  \end{itemize}
\end{ejemplo}

\vspace{1em}
Entonces, podemos decir que existen lenguajes que pueden ser finitos, o infinitos, donde los lenguajes infinitos son aquellos que son compuestos por infinitas palabras que pertenecen al lenguaje, mientras que en los finitos no.
\vspace{2em}

\hspace{1em}
En $\Sigma^*$ se definen las siguientes operaciones:
\begin{itemize}
  \item Concatenacion:
  \begin{center}
    Sean $x = a_1a_2 \dots a_n$, $y = b_1b_2 \dots b_n$, entonces:

    $x\cdot y = xy = a_1a_2 \dots a_nb_1b_2 \dots b_n$
  \end{center}
  \item Reflexión:
  \begin{center}
    Si $x = a_1a_2 \dots a_n$, entonces:

    $x^r  = a_n\dots a_2a_1$

    $x^r$ se denomina "Palabra Refleja"
  \end{center}
\end{itemize}

\section{Operaciones Con Lenguajes}


\end{document}
\documentclass[12pt]{article}
\usepackage[utf8]{inputenc}   % para tildes
\usepackage[spanish]{babel}   % español
\usepackage{color}            % colores para código
\usepackage{xcolor}           % colores en general
\definecolor{blueShyupss}{RGB}{0,0,128}
\usepackage[colorlinks=true, linkcolor=blueShyupss, urlcolor=gray, citecolor=orange]{hyperref} % para links (están con colores personalizados)
\usepackage{amsmath, amssymb} % símbolos y matemáticas
\usepackage{graphicx}         % imágenes
\usepackage{listings}         % código fuente
\usepackage{fancyhdr}         % encabezado/pie de página
\usepackage{geometry}         % márgenes
\usepackage{parskip}          % para los saltos
\usepackage{tikz}             % para automatas
\usetikzlibrary{automata, positioning} % uso específico de automatas

% definiciones para las llamadas a definiciones y teoremas"
\newtheorem{teorema}{Teorema}
\newtheorem{definicion}{Definición}
\newtheorem{lema}{Lema}
\newtheorem{demostracion}{Dem. }

% Para el texto de "Ejemplo:"
\newcommand{\ejemplo}{\vspace{1.1em}\textbf{\vspace{0.2em}\textit{Ejemplo:}} } 
\newcommand{\ejercicio}{\vspace{1.1em}\textbf{\vspace{0.2em}\textit{Ejercicio:}} } 

\geometry{margin=2cm}

% inicio del documento
\begin{document}

\title{Teoría de Automatas}
\author{Diego Soto - Universidad Austral De Chile}
\date{\today}
\maketitle

\newpage %índice automático en nueva página
\tableofcontents
\newpage

% aquí comienza el contenido del shyupss, ordenado por secciones
\section{Introducción}

La teoría de autómatas constituye uno de los pilares fundamentales de la computación teórica y el análisis formal de lenguajes. Esta disciplina permite modelar y analizar el comportamiento de sistemas computacionales mediante representaciones abstractas conocidas como autómatas. A través del estudio de estos modelos, es posible comprender mejor cómo funcionan los lenguajes formales, los compiladores, y las máquinas que los procesan. 

Para comenzar, definiremos y repasaremos algunos conceptos que son previamente sabidos del ramo Estructuras Discretas y luego de a poco, nos iremos adentrando en materia.

\section{Alfabetos y Lenguajes}

Cuando hablamos comunmente de un lenguaje, se nos puede venir a la distintas ideas, como palabras, letras, etcétera. 
Comenzaremos dando algunas definiciones de estas, para poder empezar a contruir otras fokin definiciones.

\begin{definicion}
  Un Alfabeto es un conjunto finito de símbolos. \begin{center} $\Sigma = \{a, b, c, ..., z\}$\end{center}
\end{definicion}

\begin{definicion}
  Una Palabra es una secuencia ordenada de símbolos de un alfabeto. \par \begin{center} $x = a_1a_2a_3...a_4$ \end{center}
\end{definicion}

\begin{definicion}
  El largo de una palabra es el número de símbolos que los conforman. \par \begin{center}$\left\lvert x \right\rvert  = n$\end{center}
\end{definicion}

\begin{definicion}
  La palabra vacía es una palabra que no tiene ningún símbolo.
  \begin{center}
    Palabra vacía: $\varepsilon \Rightarrow \left\lvert \varepsilon \right\rvert = 0$
  \end{center}
\end{definicion}

\begin{definicion}
  Si $\Sigma$ es un alfabeto, anotaremos como $\Sigma^*$ el conjunto de todas las palabras posibles sobre $\Sigma$.
  \begin{center}
    $\Sigma^* = \bigcup_{k\geq0} \Sigma^k = \Sigma^0 \cup \Sigma^1 \cup \Sigma^2 \cup ...$

    En donde $\Sigma^k$ es el conjunto de palabras de largo k.
  \end{center}
\end{definicion}

\begin{ejemplo}
  \begin{center}
    Si $\Sigma = \left\{a, b\right\} $, entonces: 
  
    $\Sigma^0 = \left\{\varepsilon\right\}$
  
    $\Sigma^1 = \left\{a, b\right\}$
  
    $\Sigma^2 = \left\{aa, ab, ba, bb\right\}$
  
    \dots
    
    $\Sigma^n = \left\{\epsilon, a, b, aa, ab, ba, bb, aaa, \dots\right\}$
  \end{center}
\end{ejemplo}

\begin{definicion}
  Un lenguaje es un conjunto de palabras que se pueden armar sobre un alfabeto $\Sigma$, el cual denotaremos por la letra $L$

  \begin{center}
    Con esto, tenemos que $L\subseteq \Sigma^*$
  \end{center}
\end{definicion}

\begin{ejemplo}
  \begin{itemize}
    \item $L = \{ x \in \{a, b\}^* \mid \text{x comienza con 'a'} \}$

    $L = \{a, aa, ab, aab, aba, \dots\}$
    \item $L = \{ x \in \{0, 1\}^* \mid \text{x es palindromo'} \}$
    
    $L = \{0, 1, 00, 11, 101, 010, 11011, 0110, \dots\}$
    \item $L = \{ x \in 6^+ \mid \text{Los dígitos de x suman 10}\}$
    
    $L = \{64, 22222, 235, \dots\} \text{ (Notar que las palabras acá son finitas)}$
  \end{itemize}
\end{ejemplo}

\vspace{1em}
Entonces, podemos decir que existen lenguajes que pueden ser finitos, o infinitos, donde los lenguajes infinitos son aquellos que son compuestos por infinitas palabras que pertenecen al lenguaje, mientras que en los finitos no.
\vspace{2em}

\hspace{1em}
En $\Sigma^*$ se definen las siguientes operaciones:
\begin{itemize}
  \item Concatenacion:
  \begin{center}
    Sean $x = a_1a_2 \dots a_n$, $y = b_1b_2 \dots b_n$, entonces:

    $x\cdot y = xy = a_1a_2 \dots a_nb_1b_2 \dots b_n$
  \end{center}
  \item Reflexión:
  \begin{center}
    Si $x = a_1a_2 \dots a_n$, entonces:

    $x^r  = a_n\dots a_2a_1$

    $x^r$ se denomina "Palabra Refleja"
  \end{center}
\end{itemize}

\section{Operaciones Con Lenguajes}

\newpage
\section{Recursividad Izquierda no inmediata}

Por el proceso anterior se elimina toda recursividad izquierda inmediata, no así la recursividad obtenida con derivaciones de mas de un paso. Por ejemplo en la gramática

\begin{center}
  $S \rightarrow Aa \mid b$

  $A \rightarrow Ac \mid Sd \mid e$
\end{center}

Tenemos $S \Rightarrow Aa \Rightarrow Sda$

\begin{center}
  $S \Rightarrow^* Sda$
\end{center}

\begin{lema}
  Toda L.L.C que no contiene $\varepsilon$ puede ser generado por una G.L.X sim recursividad izquierda.
\end{lema}

\begin{demostracion}
  Sea $G = (V, T, P, S)$ una G.L.C que genera L.

  \begin{enumerate}
    \item Elimina los ciclos con el procedimiento visto para verificar una gramática.
    \item Ordenar las variables $V = \{A_1, A_2, \dots, A_n \}$ de tal forma que $A_i \rightarrow A_j\alpha$ es una producción solo si $i < j$.
    
    Si $A_k \rightarrow A_j\alpha$ con $j < k$, con el procedimiento de sustitución se reemplaza en $A_k \rightarrow A_j\alpha$ por las $A_j-producciones$. Se repite este proceso.
    \item Eliminar recursividad izquierda inmediata.
  \end{enumerate}

  Se repite 1, 2 y 3 hasta que no quede recursividad izquierda.
\end{demostracion}

\begin{ejemplo}

  \begin{center}
    $S \rightarrow Aa \mid b$
  
    $A \rightarrow Ac \mid Sd \mid e$
    
  \end{center}
  \begin{enumerate}
    \item No hay ciclos
    \item $V = \{S, A\}$, $S < A$ y $A \rightarrow Sd$ con $S < A$

    Se reemplaza por las $S-producciones$

    $S \rightarrow Aa \mid b$

    $A \rightarrow Ac \mid Aad \mid bd \mid e$

    \item $S \rightarrow Aa \mid b$
    
    $A \rightarrow bdA' \mid bd \mid eA' \mid e$

    $A' \rightarrow cA' \mid c \mid adA' \mid ad$

    $V = {S, A, A'}$
  \end{enumerate}

\end{ejemplo}

\begin{ejercicio}
  Elimina la recursividad izquierda.

  $S \rightarrow aRb \mid Rb$

  $R \rightarrow ca \mid Taa$

  $T \rightarrow Ra$

  \begin{enumerate}
    \item No hay ciclos
    \item $V = {S, R, T}$
    
    $S \rightarrow aRb \mid Rb$

    $R \rightarrow ca \mid Taa$

    $T \rightarrow caa \mid Taaa$

    \item $S \rightarrow aRb \mid Rb$
    
    $R \rightarrow ca \mid Taa$

    $T \rightarrow caaT' \mid caa$

    $T' \rightarrow aaaT' \mid aaa$

    $V = {S, R, T, T'}$, $T \rightarrow cRST$
  \end{enumerate}

\end{ejercicio}

\section{Forma normal de Greibach}

Una G.L.C $G = (V, T, P, S)$ se dice qie está en Forma Normal de greibach (F.N.G) si sus producciones toenen la misma forma:

\begin{center}
    $A \rightarrow a\alpha$ con $A \in V$, $a \in T$, $\alpha \in V^*$
\end{center}

\begin{teorema}
  Toda L.L.C que no contiene $\varepsilon$ puede ser generado por una G.L.C. en su F.N.G.
\end{teorema}

\begin{demostracion}
  El algoritmo para trnasformar una G.L.C a F.N.G es:

  \begin{enumerate}
    \item Ramifica la gramática y eliminar toda recursividad izquierda.
    \item Para cada producción de la forma $A \rightarrow \alpha_1\alpha_2\dots\alpha_k$ con $k\geq2$, si $\alpha_i$, $i\geq2$ es un terminal, se reemplaza por una variable auxiliar $X_{\alpha_i}$ y se agrega la producción $X_{\alpha_i}\rightarrow\alpha_i$
    \item Se ordenan las variables $V = {A_1, A_2, \dots, A_m}$ de modo que necesariamente $A_m \rightarrow a\beta$ con $a \in T$. Por el procedimiento de sustitución, se puede sustituir en las producciones $A_i \rightarrow A_j\beta$ con $i<j$ por las $A_j - producciones$, de modo que hacer aparecer un terminal al comienzo del lado derecho, partiendo de $j = m$ hacia atrás.
  \end{enumerate}
\end{demostracion}

\begin{ejemplo}

  \begin{center}
    $S \rightarrow abA \mid Sa$
  
    $A \rightarrow Bb \mid cc$
  
    $B \rightarrow ac \mid ba$
  \end{center}
  
  \begin{enumerate}
    \item Es positiva
    \item Está ramificada

    $V = {S, A, B}$ (no hay "Recursividad izquierda no inmediata")
    
    Eliminamos "Recursividad izquierda inmediata"

    $S \rightarrow abA \mid Sa$

    $S' \rightarrow aS' \mid a$

    $A \rightarrow Bb \mid cc$

    $B \rightarrow ac \mid ba$

    \item $S \rightarrow aX_bAS' \mid aX_bA$
    
    $S' \rightarrow aS' \mid a$

    $A \rightarrow BX_b \mid cX_c$

    $B \rightarrow aX_c \mid bX_a$

    $X_a \rightarrow a$

    $X_b \rightarrow b$

    $X_c \rightarrow c$

    \item $A \rightarrow aX_cX_b \mid bX_aX_b \mid cX_c$
  \end{enumerate}

\end{ejemplo}

\end{document}